%
% Homework Details
%   - Title
%   - Due date
%   - University
%   - Class
%   - Class Alias
%   - Class Section
%   - Instructor
%   - Author
%   - AuthorID
%

\newcommand{\hmwkID}{4}
\newcommand{\hmwkTitle}{Homework\ \#\hmwkID}
\newcommand{\hmwkDueDate}{May 15, 2015 at 16:20}
\newcommand{\hmwkUniversity}{NTU}
\newcommand{\hmwkClass}{Data Structures and Algorithms}
\newcommand{\hmwkClassAlias}{DSA}
\newcommand{\hmwkClassSection}{Spring 2015}
\newcommand{\hmwkClassInstructor}{Hsuan-Tien Lin, Roger Jang}
\newcommand{\hmwkAuthorName}{Tim Liou}
\newcommand{\hmwkAuthorID}{b03902028}


\documentclass[11pt]{article}

\usepackage{fancyhdr}
\usepackage{extramarks}
\usepackage{enumitem}
\usepackage{amsmath}
\usepackage{amsthm}
\usepackage{amsfonts}
\usepackage{tikz}
\usepackage[plain]{algorithm}
\usepackage{algpseudocode}
\usepackage{listings}
\usepackage{lastpage}
\usepackage{color}

\usetikzlibrary{shapes,positioning,arrows,calc}
\usetikzlibrary{automata,positioning}

%
% Basic Document Settings
%

\topmargin=-0.45in
\evensidemargin=0in
\oddsidemargin=0in
\textwidth=6.5in
\textheight=9.0in
\headsep=0.25in

\linespread{1.1}

\pagestyle{fancy}
\lhead{\hmwkAuthorName\ (\hmwkAuthorID)}
\chead{\hmwkClassAlias\ (\hmwkUniversity, \hmwkClassSection): \hmwkTitle}
\rhead{\firstxmark}
\lfoot{\lastxmark}
\cfoot{\thepage\ of \pageref{LastPage}}

\renewcommand\headrulewidth{0.4pt}
\renewcommand\footrulewidth{0.4pt}

\setlength\parindent{0pt}
\setlength{\parskip}{1em}
%
% Create Problem Sections
%

\newcommand{\enterProblemHeader}[1]{
    \nobreak\extramarks{}{Problem \hmwkID.\arabic{#1} continued on next page\ldots}\nobreak{}
    \nobreak\extramarks{Problem \hmwkID.\arabic{#1} (continued)}{Problem \hmwkID.\arabic{#1} continued on next page\ldots}\nobreak{}
}

\newcommand{\exitProblemHeader}[1]{
    \nobreak\extramarks{Problem \hmwkID.\arabic{#1} (continued)}{Problem \hmwkID.\arabic{#1} continued on next page\ldots}\nobreak{}
    \stepcounter{#1}
    \nobreak\extramarks{Problem \hmwkID.\arabic{#1}}{}\nobreak{}
}

\setcounter{secnumdepth}{0}
\newcounter{partCounter}
\newcounter{homeworkProblemCounter}
\setcounter{homeworkProblemCounter}{1}
\nobreak\extramarks{Problem \hmwkID.\arabic{homeworkProblemCounter}}{}\nobreak{}

%
% Homework Problem Environment
%
% This environment takes an optional argument. When given, it will adjust the
% problem counter. This is useful for when the problems given for your
% assignment aren't sequential. See the last 3 problems of this template for an
% example.
%
\newenvironment{homeworkProblem}[2][-1]{
    \ifnum#1>0
        \setcounter{homeworkProblemCounter}{#1}
    \fi
    \section{\hmwkID.\arabic{homeworkProblemCounter} \hspace{0.1in} #2}
    \setcounter{partCounter}{1}
    \enterProblemHeader{homeworkProblemCounter}
}{
    \exitProblemHeader{homeworkProblemCounter}
}

%
% Title Page
%

\title{
    \vspace{2in}
    \textmd{\textbf{\hmwkClass:\ \hmwkTitle}}\\
    \normalsize\vspace{0.1in}\small{Due\ on\ \hmwkDueDate}\\
    \vspace{0.1in}\large{\textit{Instructors \hmwkClassInstructor}}
    \vspace{3in}
}

\author{
    \textbf{\hmwkAuthorName} \small{(\hmwkAuthorID)} 
}
\date{}

\renewcommand{\part}[1]{\textbf{\large Part \Alph{partCounter}}\stepcounter{partCounter}\\}

\lstset{
    language=C,
    numbers=left,
    frame=single,
    columns=fullflexible,
    basicstyle=\ttfamily
}

%
% Various Helper Commands
%

\newcommand{\subqest}[1]{ \textbf{(\arabic{partCounter})} #1 \stepcounter{partCounter} \par }

% Something need to be done
\newcommand{\pending}[1][]{~~~~\textcolor{red}{ 
        \textbf{ \large ====== Pending 
            \ifx&#1&%
            % #1 is empty
            \else
            % #1 is nonempty
            \large (#1)
            \fi
            \large ====== 
        } 
    }
}

% Useful for algorithms
\newcommand{\alg}[1]{\textsc{\bfseries \footnotesize #1}}

% For derivatives
\newcommand{\deriv}[1]{\frac{\mathrm{d}}{\mathrm{d}x} (#1)}

% For partial derivatives
\newcommand{\pderiv}[2]{\frac{\partial}{\partial #1} (#2)}

% Integral dx
\newcommand{\dx}{\mathrm{d}x}

% Alias for the Solution section header
\newcommand{\solution}{\textbf{\large Solution}}

% Probability commands: Expectation, Variance, Covariance, Bias
\newcommand{\E}{\mathrm{E}}
\newcommand{\Var}{\mathrm{Var}}
\newcommand{\Cov}{\mathrm{Cov}}
\newcommand{\Bias}{\mathrm{Bias}}



\usepackage{qtree}
\usepackage{enumitem}

\begin{document}

\pagenumbering{gobble}

\maketitle

\pagebreak

\pagenumbering{arabic}  

\begin{homeworkProblem}{Tree}
    \subqest{Do Exercise R-7.15 of the textbook.}
    \Tree [.E [ [ M F ].A U ].X N ]

    \subqest{Do Exercise R-7.24 of the textbook.}
    \begin{enumerate}[label=(\alph*)]
        \item 
            We can easily find the level of right child is larger than its parent 
            and its sibling. It implies that it will obtain the largest level if
            all the nodes (except the root) is its parent's right child. 

            \Tree [.$a_1$ (null) [.$a_2$ (null) [.$a_3$ (null) [.$\vdots$ (null) [.$a_n$ (null) ]]]]] \\

            Note that $f(a_n) = 2f(a_{n-1}) + 1$, and we can find
            \begin{align*}
                &\Rightarrow f(a_n) + 1 = 2(f(a_{n-1}) + 1) \\
                &\Rightarrow f(a_n) + 1 = 2^2(f(a_{n-2}) + 1) = \dots = 2^{n-1}(f(a_1) + 1) \\
                &\Rightarrow f(a_n) + 1 = 2^n \\
                &\Rightarrow f(a_n) = 2^n - 1\\
            \end{align*}
            We find the upper bound of level for a binary tree with $n$ nodes. 
            That is, for every node $v$ of $T$,
            \[
                f(v) \leq 2^n - 1
            \]
            
        \item 
            \Tree [.$x_1$ (null) [.$x_2$ (null) [.$x_3$ (null) [.$x_4$ (null) [.$x_5$ (null) [.$x_6$ (null) [.$x_7$ (null) ]]]]]]] \\
            There are 7 nodes in this tree. The upper bound of the level is $2^7 -1 = 127$.
            We can easily find $f(x_7) = 127$ by the formula in (a). The node $x_7$
            attains the above upper bound on $f$ in this tree.
            
    \end{enumerate}

    \subqest{Do Exercise C-7.3 of the textbook.}

    $post(v) = pre(v) + desc(v) - depth(v)$

    For the root, with post-order, we will visit its descendents first, then 
    visit it, but with pre-order, we will visit it first, then
    visit its descendents. 

    \subqest{Do Exercise C-7.7 of the textbook with pseudo code.}

    \begin{algorithm}[]
        \begin{algorithmic}[1]
            \Function{Print}{\emph{root, NumOfIndent}}
            \If{root has children}
                \State{print out the content in root and '(' with NumOfIndent indent.}
                \ForAll{$i \gets$ children}
                    \State{\Call{Print}{\emph{i, NumOfIndent + 1}}}
                \EndFor
                \State{print out ')' with NumOfIndent indent}
            \Else
                \State{print out the content in root with NumOfIndent indent.}
            \EndIf
            \State{\Return{}}
            \EndFunction{}
        \end{algorithmic}
        \caption{Print a tree with indented parenthetic representation}
    \end{algorithm}

    Call this function by \alg{Print$(root, 0)$}.

\end{homeworkProblem}


\begin{homeworkProblem}{Decision Tree}
    \subqest{Using the property that the $v_m$ values are sorted, describe an $O(M)$
    algorithm to calculate the best threshold.}

    \begin{algorithm}[]
        \begin{algorithmic}[1]
            \ForAll{$i \gets $ data}
            \If{The value of this feature is bigger than $threshold_j$}
                \State{initial $a_{j+1}$Y$b_{j+1}$N with the value in $a_j$Y$b_j$N}
                \State{$j \gets j + 1$}
            \Else
                \State{Update $a_j$Y$b_j$N.}
            \EndIf
            \EndFor
            \ForAll{$i \gets $ thresholds}
                \State{calculate the confusion by using $a_i$Y$b_i$N and $(TotalY - a_i)$Y$(TotalN - b_i)$N.}
                \State{check if this confusion is smaller than previous.}
            \EndFor
        \end{algorithmic}
        \caption{Calculate the best threshold with an $O(M)$ algorithm}
    \end{algorithm}

    The purpose of first \alg{for loop} is to calcuate and store how many Yes and
    No below each threshold. We will know the values of TotalY and TotalN
    because we can get them when we encounter the biggest thresold $v_M+1$.

    \subqest{Implement the decision tree algorithm.}
    Code part.

    \subqest{Illustrate (ideally with drawing) the internal data structure you 
    use to represent the decision tree in your memory. Please be as precise 
    as possible.}
    I use this data structure to represent the decision tree.

    \begin{lstlisting}[breaklines=true]
struct BranchChoice {
    int BestFeature;
    double BestTotalConfusion;
    double BestThresholdID;
    double BestThreshold;
};
struct DTree {
    BranchChoice Choice;
    DTree *left;  
    DTree *right;
};
    \end{lstlisting}

    \subqest{Teach your decision tree with the following examples to learn a 
    function $f$ with $\epsilon$ = 0. Draw the tree you get.}

    Left side means Yes and right side means No.

    \Tree [.Weight$\;<63.5$ -1 [ +1 [ -1 +1 ].Weight$\;<86.5$ ].Weight$\;<78.5$  ]


    \subqest{Construct your own data set with at least 2 numerical factors and
    at least 6 examples. Teach your program to make a decision tree of at
    least 2 levels with this data set. List the examples as well as draw
    the tree found. Briefly explain the tree.}
    \pending

    \subqest{Implement the random forest algorithm.}
    Code part.
\end{homeworkProblem}

\end{document}
