\documentclass[11pt]{homework}

% TODO: replace these with your information
\newcommand{\hwname}{Tim Liou}
\newcommand{\hwemail}{b03902028}
\newcommand{\hwtype}{Homework}
\newcommand{\hwnum}{1}
\newcommand{\hwclass}{Data Structures and Algorithms}
%\newcommand{\hwlecture}{1}
%\newcommand{\hwsection}{Z}

\usepackage{nicefrac}

\begin{document}
\maketitle


\renewcommand{\writtensection}[0]{1}
\question

  \begin{arabicpartsdsa}
    \item 

        Professor Donald Knuth has been called ``father of analysis of
        algorithms''. He is also the creator of \TeX\;computer typesetting system.
        He found the galley proof of his great book, The Art of Computer
        Programming, awful. Meanwhile, he saw for the first time the output of
        high-quality digital typesetting system, and decided to design his own
        typesetting system, TeX, to solve this problem. He used to pay \$2.56 to
        those who find mistakes in his books because ``256 pennies is one
        hexadecimal dollar''.

    \item 

        Proof by induction.
        \begin{induction}
            \basecase
            If string array's length is \(0\), \(i\) will not go into the while loop
            because \(i\) initialize to \(0\) and \(str[i]\)(now is \(str[0]\)) is already
            the end of the string. This algorithm then return \(i = 0\), which is the
            length of the string.
            \indhyp
            Claim: At the end of while loop, string array whose length is \(k\) return \(i = k\).
            \indstep
            Assume Claim is correct at the case that string's length is \(M\). We
            denote this string \(A\). When string's length is \(M + 1\) (we denote this
            string \(B\)), \(i\) comes to M which is A's end. However, \(str[M]\)
            isn't the end of \(B\). Therefore, \(i\) becomes to \(M + 1\), and now
            \(str[M + 1]\) is the end of \(B\). Finally, this algorithm return
            \(i = M + 1\) when the length of string is \(M + 1\).
        \end{induction}

  \end{arabicpartsdsa}

\question

  \begin{arabicpartsdsa}
    \item 

        Without loss of generality, let \(a \geq b\). We know that any number
        bigger than \(n\) cannot divide \(n\) without a remainder. Therefore, by
        definition, \(1 \leq gcd(a, b) \leq min(a, b)\). Since we choose \(i\)
        from \(min(a, b)\) to \(1\), the first \(i\) satisfing the if statement will be the
        largest one which can divide \(a\) and \(b\) without remainder. This shows that
        \(i\) is \(gcd(a, b)\).

    \item 

        This value is 1.

        When \(a = b \times k, k > 1, k \in N\), we have the best case.

  \end{arabicpartsdsa}

\pagebreak

\question

  \begin{arabicpartsdsa}
    \item 

        \begin{math}
            a = l \times k \Rightarrow k = \frac{a}{l} \\
            b = l \times j \Rightarrow j = \frac{b}{l} \\
        \end{math}

        By definition,

        \begin{math}
            k = gcd(k, j) \times r, j = gcd(k, j) \times w,\ which \ gcd(r, w) = 1. \\
            a = l \times gcd(\frac{a}{l}, \frac{b}{l}) \times r  \\
            b = l \times gcd(\frac{a}{l}, \frac{b}{l}) \times w  \\
        \end{math}

        Claim: \(gcd(a, b) = l \times gcd(\frac{a}{l}, \frac{b}{l}) \)

        \begin{arabicpartsdsasec}
        \item 

            We can easily observe that \(a\ mod\ (l \times gcd(\frac{a}{l}, \frac{b}{l})) = 0 \) and \(b\ mod\ (l \times gcd(\frac{a}{l}, \frac{b}{l})) = 0\)

        \item 

            Suppose \(gcd(a, b)\) is greater than \(gcd(a, b) = l \times gcd(\frac{a}{l}, \frac{b}{l}) \) and denote it 
            \((l \times gcd(\frac{a}{l}, \frac{b}{l})\times c)\) which \(c \geq 2\). 

            \begin{math}
                \Rightarrow 
                \begin{cases}
                a = l \times gcd(\frac{a}{l}, \frac{b}{l}) \times r = l \times gcd(\frac{a}{l}, \frac{b}{l}) \times r \times c \times A \) \\
                b = l \times gcd(\frac{a}{l}, \frac{b}{l}) \times w = l \times gcd(\frac{a}{l}, \frac{b}{l}) \times r \times c \times B \) \\
                \end{cases} \\
                \Rightarrow 
                \begin{cases}
                    r = c \times A  \\
                    w = c \times B  \\
                \end{cases} \\
                \Rightarrow gcd(r, w) \geq c \geq 2 \\
            \end{math} \\
            This is a contradiction. Therefore,
            \[gcd(a, b) = l \times gcd(\frac{a}{l}, \frac{b}{l}) \] \\

        \end{arabicpartsdsasec}

    \item 

        This value is \(1\), If the pseudo code enter that line, this means that the
        number from \(min(a, b)\) to \(2\) couldn't divide \(a\), \(b\) without remainder. By
        1.2(1), we know that \(1 \leq gcd(a, b) \leq min(a, b)\).
        Therefore \(1\) is the only value can return.

    \item 

        This value must be 2. If we choose 1, the if statement will always be
        true, this means that we will always return \(1 \times gcd(a, b)\) and never
        compute the answer. If we choose the number greater than 2, we will miss
        if a \(mod\) \(2 = 0\) and b \(mod\) \(2 = 0\) because for loop start from the number
        greater than 2.

  \end{arabicpartsdsa}

\pagebreak

\question

  \begin{arabicpartsdsa}
    \item 

        \begin{verbatim}
        n = 14     m = 56     ans =  1
        ------------------------------
        n =  7     m = 21     ans =  2

        n =  7     m = 14     ans =  2

        n =  7     m =  0     ans =  2
        ------------------------------
        n =  7     m =  0     ans = 14 
        \end{verbatim}

    \item 

        Claim: \(m\) will finally become 0.
        \begin{arabicpartsdsasec}
        \item
            At the end of each iteration, m is getting smaller and smaller.  \\
            Case 1: if \(n > m\), \(n\) will swap with \(m\). Therefore \(m\) becomes smaller.   \\
            Case 2: if \(n = m\), \(m\) will become 0. \(m\) becomes smaller and is 0.  \\
            Case 3: if \(n < m\), \(m\) will become \((m-n)\). \(m\) becomes smaller. 
        \item
            We can easily observe that \(n\) will never become to 0.
        \item
            If we enter the line \(m \leftarrow (m - n)\), \(m \geq n\), \(m - n \geq 0\), this shows that \(m\) will getting
            smaller but still greater than 0 if \(m \neq n\). We notice that \(m\) will not have a chance to become 0
            before enter this statement, this means that \(m \geq n > 0\).\\
            Case 1: if \(n = m \neq 1\), \(m\) will become 0.  \\
            Case 2: Since \(m\) is getting smaller, \(m\) will finally become 1. \(m \geq n >0\), \(n\) will become 1, too. \(m = m - n = 0\).

        \end{arabicpartsdsasec}
        This prove that \(m\) will finally become 0, that is, GCD-By-Binary satisfies the finiteness
        property.

    \item

        By definition,  \\
        \(a = gcd(a, b) \times k, b = gcd(a, b) \times r, gcd(k, r) = 1,  k > r \geq 1 \) \\
        \(\Rightarrow a - b = gcd(a, b)\times(k - r)\) \\ \\
        Case 1: r = 1 \\
        \(a - b = gcd(a, b)\times(k - 1)\)  \\
        \(gcd(a - b, b) = gcd(a, b)\times gcd(k-1, 1) = gcd(a, b),\) by 1.3(1). \\ \\
        Case 2: r > 1 \\
        \(a - b = gcd(a, b)\times(k - r)\) \\
        \(gcd(a - b, b) = gcd(a, b)\times gcd(k-r, r)\) \\
        if \(gcd(k-r, r) = c \geq 2. k - r = c \times A, r = c \times B\) \\
        \(\Rightarrow k = c \times (A+B), r = c \times B\) \\
        \(\Rightarrow gcd(k, r) \geq c \geq 2 \)
        This is a contradiction. Therefore, 
        \(gcd(k-r, r) = 1\) \\
        \(\Rightarrow gcd(a-b, b) = gcd(a, b).\) \\

    \item

        4 \times $\lceil \log_2 a \rceil$ + 1

        The worst case is that \(a\) take 2 iterations to become half of itself, and so does \(b\).
        This implies that it take at most 4 \times $\lceil \log_2 a \rceil$ iterations. If they both
        become 1, it take one more iteration. Therefore, the tightest upper bound I can think of is 
        4 \times $\lceil \log_2 a \rceil$ + 1.

    \end{arabicpartsdsa}

\question

    \begin{arabicpartsdsa}
    \item

        \begin{verbatim}
        m = 56     n = 14     tmp =  *
        ------------------------------

        ------------------------------
        m = 56     n = 14     tmp =  *
        \end{verbatim}

    \item

        Claim1: \(3a\) \(mod\) \(3b = 0\) if \(a\) \(mod\) \(b = 0\)\\
        \(a\ mod\ b = 0 \Rightarrow a = b \times r, r > 0\)\\
        \(\Rightarrow 3a = 3b \times r \)\\
        \(\Rightarrow3a\ mod\ 3b = 0 \)

        Claim2: \(3a\ mod\ 3b = 3k\ if\ a\ mod\ b = k\)\\
        \(a\ mod\ b = k \Rightarrow a = b \times r + k, 0 < k < b \)\\
        \(\Rightarrow 3a = 3b \times r + 3k, 0 < 3k < 3b\)\\
        \(\Rightarrow 3a\ mod\ 3b = 3k\)

        In while loop, we check \(m\ mod\ n\) is 0 or not\\
        Case 1: \(m\ mod\ n = 0\), by Claim1, we know that \(3m\ mod\ 3n = 0\), which will be out of while loop.\\
        Case 2: \(m\ mod\ n = k \neq 0\), \(m\) will become \(n\), and \(n\) will become \(k\). By Claim2, we know that \(3m\ mod\ 3n = 3k\), 
        then \(3m\) will become \(3n\), and \(3n\) will become \(3k\).

        This imply that gcd-by-euclid\((3a, 3b)\) and gcd-by-euclid\((a, b)\) take the same iterations.
    \item

        Proof by induction.
        \begin{induction}
            \basecase
            If \(T = 0\), this means that \(a\ mod\ b = 0\), and we know that \(a > b\), this implies
            that \(a > b \geq 1\). We have \(a > b \geq 1 = F_2 = F_1\)
            \indhyp
            Claim: If applying GCD-By-Euclid(a, b) takes T iterations, we can show that
            \(a \geq F_{T+2}\) and \(b \geq F_{T+1}\)
            \indstep
            Assume Claim is correct when \(T = k\), this means that \(a_k \geq F_{T+2}\) and \(b_k \geq F_{T+1}\)

            When \(T = k+1\), We can easily find that 
            \[a_{k+1} = b_{k+1} \times l + (a_{k+1}\ mod\ b_{k+1}) \geq b_{k+1} + (a_{k+1}\ mod\ b_{k+1}) \]
            Notice that \(gcd(a_{k+1}, b_{k+1}) = gcd(b_{k+1}, a_{k+1}\ mod\ b_{k+1})\),
            and we now remain k iterations, therefore, we have 
            \[b_{k+1} \geq F_{k+2} = F_{(k+1)+1}\] 
            \[(a_{k+1}\ mod\ b_{k+1}) \geq F_{k+1}\]
            Then, we also get 
            \[a_{k+1} \geq b_{k+1} + (a_{k+1}\ mod\ b_{k+1}) \geq F_{k+2} + F_{k+1} = F_{k+3} = F_{(k+1)+2}\]

        \end{induction}
        We conclude that the Claim is correct.

    \end{arabicpartsdsa}


\question

  \begin{arabicpartsdsa}
  \item 

      code done.

  \item
      Result: \\
      \begin{verbatim}
      Average-GCD-By-Reverse-Search : 11254
      Average-GCD-By-Filter         : 6491
      Average-GCD-By-Filter-Faster  : 6491 
      Average-GCD-By-Binary         : 18
      Average-GCD-By-Euclid         : 8
      \end{verbatim}

      GCD-By-Euclid and GCD-By-Binary take fewer iterations than the others. This means they are much faster.
      On average, GCD-By-Filter-Faster is not much faster than GCD-By-Filter in this case because \(11260 = 2^2 \times 5 \times 563\),
      which GCD-By-Filter-Faster cannot save lots of iterations.
  \end{arabicpartsdsa}
\end{document}

