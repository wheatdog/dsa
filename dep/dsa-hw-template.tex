\documentclass[11pt]{article}

\usepackage{fancyhdr}
\usepackage{extramarks}
\usepackage{amsmath}
\usepackage{amsthm}
\usepackage{amsfonts}
\usepackage{tikz}
\usepackage[plain]{algorithm}
\usepackage{algpseudocode}
\usepackage{listings}
\usepackage{lastpage}

\usetikzlibrary{automata,positioning}

%
% Basic Document Settings
%

\topmargin=-0.45in
\evensidemargin=0in
\oddsidemargin=0in
\textwidth=6.5in
\textheight=9.0in
\headsep=0.25in

\linespread{1.1}

\pagestyle{fancy}
\lhead{\hmwkAuthorName\ (\hmwkAuthorID)}
\chead{\hmwkClassAlias\ (\hmwkUniversity, \hmwkClassSection): \hmwkTitle}
\rhead{\firstxmark}
\lfoot{\lastxmark}
\cfoot{\thepage\ of \pageref{LastPage}}

\renewcommand\headrulewidth{0.4pt}
\renewcommand\footrulewidth{0.4pt}

\setlength\parindent{0pt}

%
% Homework Details
%   - Title
%   - Due date
%   - University
%   - Class
%   - Class Alias
%   - Class Section
%   - Instructor
%   - Author
%   - AuthorID
%

\newcommand{\hmwkID}{2}
\newcommand{\hmwkTitle}{Homework\ \#\hmwkID}
\newcommand{\hmwkDueDate}{April 14, 2015 at 16:20}
\newcommand{\hmwkUniversity}{NTU}
\newcommand{\hmwkClass}{Data Structures and Algorithms}
\newcommand{\hmwkClassAlias}{DSA}
\newcommand{\hmwkClassSection}{Spring 2015}
\newcommand{\hmwkClassInstructor}{Hsuan-Tien Lin, Roger Jang}
\newcommand{\hmwkAuthorName}{Tim Liou}
\newcommand{\hmwkAuthorID}{b03902028}

%
% Create Problem Sections
%

\newcommand{\enterProblemHeader}[1]{
    \nobreak\extramarks{}{Problem \hmwkID.\arabic{#1} continued on next page\ldots}\nobreak{}
    \nobreak\extramarks{Problem \hmwkID.\arabic{#1} (continued)}{Problem \hmwkID.\arabic{#1} continued on next page\ldots}\nobreak{}
}

\newcommand{\exitProblemHeader}[1]{
    \nobreak\extramarks{Problem \hmwkID.\arabic{#1} (continued)}{Problem \hmwkID.\arabic{#1} continued on next page\ldots}\nobreak{}
    \stepcounter{#1}
    \nobreak\extramarks{Problem \hmwkID.\arabic{#1}}{}\nobreak{}
}

\setcounter{secnumdepth}{0}
\newcounter{partCounter}
\newcounter{homeworkProblemCounter}
\setcounter{homeworkProblemCounter}{1}
\nobreak\extramarks{Problem \hmwkID.\arabic{homeworkProblemCounter}}{}\nobreak{}

%
% Homework Problem Environment
%
% This environment takes an optional argument. When given, it will adjust the
% problem counter. This is useful for when the problems given for your
% assignment aren't sequential. See the last 3 problems of this template for an
% example.
%
\newenvironment{homeworkProblem}[2][-1]{
    \ifnum#1>0
        \setcounter{homeworkProblemCounter}{#1}
    \fi
    \section{\hmwkID.\arabic{homeworkProblemCounter} \hspace{0.1in} #2}
    \setcounter{partCounter}{1}
    \enterProblemHeader{homeworkProblemCounter}
}{
    \exitProblemHeader{homeworkProblemCounter}
}

%
% Title Page
%

\title{
    \vspace{2in}
    \textmd{\textbf{\hmwkClass:\ \hmwkTitle}}\\
    \normalsize\vspace{0.1in}\small{Due\ on\ \hmwkDueDate}\\
    \vspace{0.1in}\large{\textit{Instructors \hmwkClassInstructor}}
    \vspace{3in}
}

\author{
    \textbf{\hmwkAuthorName} \small{(\hmwkAuthorID)} 
}
\date{}

\renewcommand{\part}[1]{\textbf{\large Part \Alph{partCounter}}\stepcounter{partCounter}\\}

\lstset{
    language=C,
    numbers=left,
    frame=single,
    columns=fullflexible,
    basicstyle=\ttfamily
}

%
% Various Helper Commands
%

% Something need to be done
\newcommand{\pending}{~~~~\textbf{\large ====== Pending ======}}

% Useful for algorithms
\newcommand{\alg}[1]{\textsc{\bfseries \footnotesize #1}}

% For derivatives
\newcommand{\deriv}[1]{\frac{\mathrm{d}}{\mathrm{d}x} (#1)}

% For partial derivatives
\newcommand{\pderiv}[2]{\frac{\partial}{\partial #1} (#2)}

% Integral dx
\newcommand{\dx}{\mathrm{d}x}

% Alias for the Solution section header
\newcommand{\solution}{\textbf{\large Solution}}

% Probability commands: Expectation, Variance, Covariance, Bias
\newcommand{\E}{\mathrm{E}}
\newcommand{\Var}{\mathrm{Var}}
\newcommand{\Cov}{\mathrm{Cov}}
\newcommand{\Bias}{\mathrm{Bias}}


