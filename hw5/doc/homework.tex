%
% Homework Details
%   - Title
%   - Due date
%   - University
%   - Class
%   - Class Alias
%   - Class Section
%   - Instructor
%   - Author
%   - AuthorID
%

\newcommand{\hmwkID}{5}
\newcommand{\hmwkTitle}{Homework\ \#\hmwkID}
\newcommand{\hmwkDueDate}{May 26, 2015 at 16:20}
\newcommand{\hmwkUniversity}{NTU}
\newcommand{\hmwkClass}{Data Structures and Algorithms}
\newcommand{\hmwkClassAlias}{DSA}
\newcommand{\hmwkClassSection}{Spring 2015}
\newcommand{\hmwkClassInstructor}{Hsuan-Tien Lin, Roger Jang}
\newcommand{\hmwkAuthorName}{Tim Liou}
\newcommand{\hmwkAuthorID}{b03902028}

\input{../../dep/dsa-hw-template.tex}

\usepackage{qtree}

\begin{document}

\pagenumbering{gobble}

\maketitle

\pagebreak

\pagenumbering{arabic}  

\begin{homeworkProblem}{Heap and Hash}
    \subqest{Complete Exercise R-8.24 of the textbook.}
    Consider a min heap. The insertion of an entry with key 32 would cause up-head
    bubbling to proceed all the way up to a child of the root in this heap.

    \Tree [.1 [.3 [.5 [.9 17 19 ] [.11 21 23 ] ] [.7 [.13 25 27 ] [.15 29 31 ] ] ] [.33 [.47 [.55 57 59 ] [.49 53 51 ] ] [.35 [.41 45 43 ] [.37 39 (32) ] ] ] ]

    \subqest{Complete Exercise C-8.4 of the textbook.}
    Combine each key we want to insert to the stack with a priority variable.
    These combinations would be insert to the MaxPriorityQueue which compare
    elements by the priority variable. Note that the basic idea of this method
    is the fact that every new element is pushed in stack with a priority higher
    than the current one. Then, MaxPriorityQueue will behave in the LIFO way, 
    which is what we want.

    \begin{lstlisting}[breaklines=true]
    class Stack {
        class Element { int priority, Key element; };
        MaxPriorityQueue<Element> MaxPQ;
        int top_priority = 0;

        void push(Key key) { MaxPQ.insert(Element(top_priority++, key)); }
        void pop() { top_priority--; MaxPQ.removeMax() }
        const Key& top() { return MaxPQ.max(); }
        int size() { return top_priority; }
        bool empty() { return (top_priority == 0); }
    };
    \end{lstlisting}

    \pagebreak

    \subqest{Complete Exercise C-8.14 of the textbook.}

    \begin{algorithm}[]
        \begin{algorithmic}[1]
            \Function{FindLeqElemInHeap}{\emph{heap, key}}
            \While{heap.top() $\leq$ key}
                \State{insert heap.top() into the output list, and do heap.removeMin()}
            \EndWhile
            \State{\Return the output list}
            \EndFunction{}
        \end{algorithmic}
        \caption{Compute all the entries in a heap with a key less than or equal to the value}
    \end{algorithm}

    \subqest{Hash function is everywhere. Use any search engine to study the 
        term “MinHash” Explain to the TAs what it is and why it is useful. Also,
        cite the website that you learn the term from.}
        I learn MinHash from 
        \alg{http://shihpeng.org/tag/minhash/} and
        \alg{http://en.wikipedia.org/wiki/MinHash}.

        MinHash is a technique to estimate how similar two sets are. We can use
        Jaccard similarity coefficient to find out the similarity between two
        sets.
        \begin{align*}
            J(A,B) = \frac{\left|A \cap B\right|}{\left|A \cup B\right|}
        \end{align*}
        However, it is slow to compute $J(A,B)$ by the the way to check every
        elements in this two sets. MinHash was designed to compute $J(A,B)$ fast.
 
        Consider a case, $S_1=\left\{\,b,c\,\right\}$, $S_2=\left\{\,a,b\,\right\}$,
        $S_1=\left\{\,b\,\right\}$
        \begin{center}
            \begin{tabular}{  c | c | c | c }
                Element & $S_1$ & $S_2$ & $S_3$ \\ \hline \hline
                a & 0 & 1 & 0\\
                b & 1 & 1 & 1\\
                c & 1 & 0 & 0\\
                d & 0 & 0 & 0\\
            \end{tabular}
        \end{center}
        And we permutate the element,
        \begin{center}
            \begin{tabular}{  c | c | c | c }
                Element & $S_1$ & $S_2$ & $S_3$ \\ \hline \hline
                c & 1 & 0 & 0\\
                d & 0 & 0 & 0\\
                b & 1 & 1 & 1\\
                a & 0 & 1 & 0\\
            \end{tabular}
        \end{center}
        Let $h^k_{min}$ be a hash function, which $k$ is a permutation of element.
        In this case, $k$ is $cdba$. And $h^k_{min}$ is defined as the symbol of
        the first non-zero row. In this case, $h^{cdba}_{min}(S_1) = c$ ,
        $h^{cdba}_{min}(S_2) = b$ $h^{cdba}_{min}(S_2) = a$. We can find that
        the value of $J(A,B)$ is as same as the possibility of 
        $h^k_{min}(A) = h^k_{min}(B)$. The following is the proof.

        For any two sets, we have three possible result for every elements.
        \begin{description}
            \item[Type 1] both are zero, that is, both sets do not contain this element.
            \item[Type 2] one is zero and the other is one, that is, one set
                contains this element.
            \item[Type 3] both are one, that is, both sets contain this element.
        \end{description}
        The number of Type 1 is $M_1$, the number of Type 2 is $M_2$, and
        the number of Type 3 is $M_3$. 
        \begin{align*}
        J(A,B) = \frac{M_3}{M_3 + M_2}
        \end{align*}
        The possibility of $h^k_{min}(A) = h^k_{min}(B)$ is 
        \begin{align*}
            P(\text{Type 3} \,|\, (\text{Type 2 or Type 3})) &= \frac{\frac{M_3}{M_1 + M_2 + M_3}}{\frac{M_2 + M_3}{M_1 + M_2 + M_3}} \\
            &= \frac{M_3}{M_3 + M_2} = J(A,B)
        \end{align*}

        If we choose enough number of $k$, the possibility of 
        $h^k_{min}(A) = h^k_{min}(B)$ will be closer to $J(A,B)$.

        This method would be more efficient than the naive algorithm. The time 
        complexity change from $O(M)$ to $O(1)$ since we don't have to go through
        all elements. This could be useful for the sets contain lots of elements.


    \subqest{Describe an algorithm to find out the position that the two strings
        differ efficiently. Briefly discuss and justify the time complexity of
        your algorithm.}
    \begin{algorithm}[]
        \begin{algorithmic}[1]
            \Function{FindDiffPostion}{\emph{s1, s2}}
                \State{start $\gets$ 0}
                \State{end $\gets$ the length of string - 1}
                \State{middle $\gets$ $\lfloor$(start + end) / 2$\rfloor$}
                \While{start $\neq$ end}
                    \If{\Call{postfixHash}{s1, end - middle} $\neq$ \Call{postfixHash}{s2, end - middle}}
                        \State{start $\gets$ middle + 1}
                        \State{middle $\gets$ $\lfloor$(start + end) / 2$\rfloor$}
                    \Else
                        \State{end $\gets$ middle}
                        \State{middle $\gets$ $\lfloor$(start + end) / 2$\rfloor$}
                    \EndIf
                \EndWhile
                \State{\Return{start}}
            \EndFunction{}
        \end{algorithmic}
        \caption{Find the position that the two strings differ efficiently}
    \end{algorithm}
    The time complexity of this algorithm is $O(\log n)$, which $n$ is the length
    of two strings, because we check whether the hash values of the last half of 
    two strings are the same. If they are same, it means the postion that two 
    strings differ is in first half. Otherwise, the position is in the last half.
    After one loop in \alg{while}, the length of string we need to check becomes
    half. This implies that the time complexity of this algorithm is $O(\log n)$.


    \subqest{Construct a perfect hash function that is efficiently computable
            for the following 32 standard keywords in C. You need to explain why
            the hash function is perfect and why it is efficiently computable to
            get the full bonus.}
        Construct a simple hash function which would not make the values of 
        these 32 keywords collision. For example, 
        $h(s) = s.front() + 27 \times s.back() + 27^2 \times s.length()$.
        It can show that these hashcodes are not the same.
        \lstinputlisting[showstringspaces=false]{../code/hashcode.example}
        Then, call these hashcodes $S = \left\{\,a_1, a_2, \hdots, a_{32}\,\right\}$.
        Note that the values in $S$ are sorted. Use binary search to find the value
        in $S$, and the perfect hashcode is the order number of the value. For 
        instance, the hash function value of the keyword 'struct' is 7621, then
        the perfect hashcode is 26.

        This method is efficiently because we can build the table in advance. 
        Besides, by using this method, we can change types of objects which
        are hard to searched by binary search to number, which is easy to be 
        searched by binary search.
       

        


            

\end{homeworkProblem}

\begin{homeworkProblem}{Distributed System}
    \subqest{Finish/rewrite the BinomialHeap class, and describe how you test 
        whether the data structure is correct.}
        I wrote a simple cpp file to test this data structure. 
        \lstset{breaklines=true}
        \lstinputlisting[language=c++,showstringspaces=false]{../code/bhtest.cpp}
        
        I can check the correctness of this data structure by the output.
        \lstinputlisting[showstringspaces=false]{../code/bhtest.out}


\end{homeworkProblem}
\end{document}
