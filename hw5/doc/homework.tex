%
% Homework Details
%   - Title
%   - Due date
%   - University
%   - Class
%   - Class Alias
%   - Class Section
%   - Instructor
%   - Author
%   - AuthorID
%

\newcommand{\hmwkID}{5}
\newcommand{\hmwkTitle}{Homework\ \#\hmwkID}
\newcommand{\hmwkDueDate}{May 26, 2015 at 16:20}
\newcommand{\hmwkUniversity}{NTU}
\newcommand{\hmwkClass}{Data Structures and Algorithms}
\newcommand{\hmwkClassAlias}{DSA}
\newcommand{\hmwkClassSection}{Spring 2015}
\newcommand{\hmwkClassInstructor}{Hsuan-Tien Lin, Roger Jang}
\newcommand{\hmwkAuthorName}{Tim Liou}
\newcommand{\hmwkAuthorID}{b03902028}


\input{../../dep/dsa-hw-template.tex}

\begin{document}

\pagenumbering{gobble}

\maketitle

\pagebreak

\pagenumbering{arabic}  

\begin{homeworkProblem}{Heap and Hash}
    \subqest{Complete Exercise R-8.24 of the textbook.}
    \pending

    \subqest{Complete Exercise C-8.4 of the textbook.}
    \pending

    \subqest{Complete Exercise C-8.14 of the textbook.}
    \pending

    \subqest{Hash function is everywhere. Use any search engine to study the 
        term “MinHash” Explain to the TAs what it is and why it is useful. Also,
        cite the website that you learn the term from.}
    \pending

    \subqest{Describe an algorithm to find out the position that the two strings
        differ efficiently. Briefly discuss and justify the time complexity of
        your algorithm.}
    \pending

    \subqest{Construct a perfect hash function that is efficiently computable
            for the following 32 standard keywords in C. You need to explain why
            the hash function is perfect and why it is efficiently computable to
            get the full bonus.}
    \pending

\end{homeworkProblem}

\begin{homeworkProblem}{Distributed System}
    \subqest{Finish/rewrite the BinomialHeap class, and describe how you test 
        whether the data structure is correct.}
    \pending

    \subqest{Implement the system for three kinds of commands below.}
    \pending[code part]


\end{homeworkProblem}
\end{document}
