%
% Homework Details
%   - Title
%   - Due date
%   - University
%   - Class
%   - Class Alias
%   - Class Section
%   - Instructor
%   - Author
%   - AuthorID
%

\newcommand{\hmwkID}{2}
\newcommand{\hmwkTitle}{Homework\ \#\hmwkID}
\newcommand{\hmwkDueDate}{April 14, 2015 at 16:20}
\newcommand{\hmwkUniversity}{NTU}
\newcommand{\hmwkClass}{Data Structures and Algorithms}
\newcommand{\hmwkClassAlias}{DSA}
\newcommand{\hmwkClassSection}{Spring 2015}
\newcommand{\hmwkClassInstructor}{Hsuan-Tien Lin, Roger Jang}
\newcommand{\hmwkAuthorName}{Tim Liou}
\newcommand{\hmwkAuthorID}{b03902028}

\documentclass[11pt]{article}

\usepackage{fancyhdr}
\usepackage{extramarks}
\usepackage{enumitem}
\usepackage{amsmath}
\usepackage{amsthm}
\usepackage{amsfonts}
\usepackage{tikz}
\usepackage[plain]{algorithm}
\usepackage{algpseudocode}
\usepackage{listings}
\usepackage{lastpage}
\usepackage{color}

\usetikzlibrary{shapes,positioning,arrows,calc}
\usetikzlibrary{automata,positioning}

%
% Basic Document Settings
%

\topmargin=-0.45in
\evensidemargin=0in
\oddsidemargin=0in
\textwidth=6.5in
\textheight=9.0in
\headsep=0.25in

\linespread{1.1}

\pagestyle{fancy}
\lhead{\hmwkAuthorName\ (\hmwkAuthorID)}
\chead{\hmwkClassAlias\ (\hmwkUniversity, \hmwkClassSection): \hmwkTitle}
\rhead{\firstxmark}
\lfoot{\lastxmark}
\cfoot{\thepage\ of \pageref{LastPage}}

\renewcommand\headrulewidth{0.4pt}
\renewcommand\footrulewidth{0.4pt}

\setlength\parindent{0pt}
\setlength{\parskip}{1em}
%
% Create Problem Sections
%

\newcommand{\enterProblemHeader}[1]{
    \nobreak\extramarks{}{Problem \hmwkID.\arabic{#1} continued on next page\ldots}\nobreak{}
    \nobreak\extramarks{Problem \hmwkID.\arabic{#1} (continued)}{Problem \hmwkID.\arabic{#1} continued on next page\ldots}\nobreak{}
}

\newcommand{\exitProblemHeader}[1]{
    \nobreak\extramarks{Problem \hmwkID.\arabic{#1} (continued)}{Problem \hmwkID.\arabic{#1} continued on next page\ldots}\nobreak{}
    \stepcounter{#1}
    \nobreak\extramarks{Problem \hmwkID.\arabic{#1}}{}\nobreak{}
}

\setcounter{secnumdepth}{0}
\newcounter{partCounter}
\newcounter{homeworkProblemCounter}
\setcounter{homeworkProblemCounter}{1}
\nobreak\extramarks{Problem \hmwkID.\arabic{homeworkProblemCounter}}{}\nobreak{}

%
% Homework Problem Environment
%
% This environment takes an optional argument. When given, it will adjust the
% problem counter. This is useful for when the problems given for your
% assignment aren't sequential. See the last 3 problems of this template for an
% example.
%
\newenvironment{homeworkProblem}[2][-1]{
    \ifnum#1>0
        \setcounter{homeworkProblemCounter}{#1}
    \fi
    \section{\hmwkID.\arabic{homeworkProblemCounter} \hspace{0.1in} #2}
    \setcounter{partCounter}{1}
    \enterProblemHeader{homeworkProblemCounter}
}{
    \exitProblemHeader{homeworkProblemCounter}
}

%
% Title Page
%

\title{
    \vspace{2in}
    \textmd{\textbf{\hmwkClass:\ \hmwkTitle}}\\
    \normalsize\vspace{0.1in}\small{Due\ on\ \hmwkDueDate}\\
    \vspace{0.1in}\large{\textit{Instructors \hmwkClassInstructor}}
    \vspace{3in}
}

\author{
    \textbf{\hmwkAuthorName} \small{(\hmwkAuthorID)} 
}
\date{}

\renewcommand{\part}[1]{\textbf{\large Part \Alph{partCounter}}\stepcounter{partCounter}\\}

\lstset{
    language=C,
    numbers=left,
    frame=single,
    columns=fullflexible,
    basicstyle=\ttfamily
}

%
% Various Helper Commands
%

\newcommand{\subqest}[1]{ \textbf{(\arabic{partCounter})} #1 \stepcounter{partCounter} \par }

% Something need to be done
\newcommand{\pending}[1][]{~~~~\textcolor{red}{ 
        \textbf{ \large ====== Pending 
            \ifx&#1&%
            % #1 is empty
            \else
            % #1 is nonempty
            \large (#1)
            \fi
            \large ====== 
        } 
    }
}

% Useful for algorithms
\newcommand{\alg}[1]{\textsc{\bfseries \footnotesize #1}}

% For derivatives
\newcommand{\deriv}[1]{\frac{\mathrm{d}}{\mathrm{d}x} (#1)}

% For partial derivatives
\newcommand{\pderiv}[2]{\frac{\partial}{\partial #1} (#2)}

% Integral dx
\newcommand{\dx}{\mathrm{d}x}

% Alias for the Solution section header
\newcommand{\solution}{\textbf{\large Solution}}

% Probability commands: Expectation, Variance, Covariance, Bias
\newcommand{\E}{\mathrm{E}}
\newcommand{\Var}{\mathrm{Var}}
\newcommand{\Cov}{\mathrm{Cov}}
\newcommand{\Bias}{\mathrm{Bias}}





\begin{document}

\pagenumbering{gobble}

\maketitle

\pagebreak

\pagenumbering{arabic}  
\begin{homeworkProblem}{More about c++}
    \textbf{(1)} The \alg{sub1} may result in a run-time error. Why?
    \\

    This function return a reference to a local varible, whose lifetime limited to
    the scope of the function call. Once \alg{sub1} return, \alg{int c} is dead. 
    Reference to a dead object is useless. Therefore, it would result in a run-time
    error.
    \\

    \textbf{(2)} The \alg{sub2} does not result in a run-time error,
    but there may be some other problem. What is the problem?
    \\

    \alg{int *pc} is allocated on the heap memory not on the stack memory, that is,
    its lifetime doesn't limited to the function call. Therefore, this function works.
    However, clients who call this function need to free memory by themselves. That is
    annoying since clients do not allocte anything but have to free something out. If
    clients forget to free memory, terrible memory leak occurs.
    \\
    
\end{homeworkProblem}

\begin{homeworkProblem}{Arrays, Linked List, and Recursion}
    \textbf{(1)} Do Exercise C-3.4 of the textbook. (The faster the better!)
    \\

    Sort the array $A$ first, and go through the whole array to find which values are as 
    same as their neighbors.
    \\

    \begin{algorithm}[]
        \begin{algorithmic}[1]
            \State{\Call{Quick-Sort}{$A$}}
            \State{$i \gets 1$}
            \State{$old \gets A[0]$}
            \While{$i \neq n$}
                \If{$old$ is equal to $A[i]$}
                \State{$old$ is one of the $repeated~integers$}
                \EndIf
                \State{$old \gets A[i]$}
                \State{$i \gets i + 1$}
            \EndWhile
        \end{algorithmic}
        \caption{Find the repeated integers}
    \end{algorithm}

    The time complexity of this \alg{while loop} is $O(n)$ and the time complexity of
    \alg{Quick-Sort} is $O(n \log n)$. Therefore, the time complexity of this algorithm
    is $O(n \log n)$.
    \\

    \pagebreak

    \textbf{(2)} Describe the memory layout and the function for getting/putting values
    from/to the matrix.
    \\

    \[
    A_n = \bordermatrix{~      & 0       & 1       & 2       & \cdots & n - 2   & n - 1    \cr
                        0      & a_{0,0} & 0       & 0       & \cdots & 0       & 0        \cr
                        1      & a_{1,0} & a_{1,1} & 0       & \cdots & 0       & 0        \cr
                        2      & a_{2,0} & a_{2,1} & a_{2,2} & \cdots & 0       & 0        \cr
                        \vdots & \vdots  & \vdots  & \vdots  & \ddots & \vdots  & \vdots   \cr
                        n - 2  & a_{n-2,0} & a_{n-2,1} & a_{n-2,2} & \cdots & a_{n-2,n-2} & 0        \cr
                        n - 1  & a_{n-1,0} & a_{n-1,1} & a_{n-1,2} & \cdots & a_{n-1,n-2} & a_{n-1,n-1}  \cr}
    \]



    We store $a_{i,j}$ with a dense one-dimensional array B at the position of 
    $B[(\sum\limits_{k=0}^i k) + j] = B[\frac{i(i+1)}{2} + j]$. 

    \[
        \bordermatrix{B[~]   & 0       & 1       & 2       & 3       & 4       & 5       & \cdots   \cr
                      A      & a_{0,0} & a_{1,0} & a_{1,1} & a_{2,0} & a_{2,1} & a_{2,2} & \cdots   \cr}
    \]

    Functions below show how to input or output the value in this data structure. 

    \begin{algorithm}[]
        \begin{algorithmic}[1]
            \Function{Get}{$B, row, col$}
                \State{\Return{$B[\frac{row(row+1)}{2} + col]$}}
            \EndFunction{}
        \end{algorithmic}
        \caption{Get the value from this matrix}
    \end{algorithm}

    \begin{algorithm}[]
        \begin{algorithmic}[1]
            \Function{Put}{$B, row, col, value$}
                \State{$B[\frac{row(row+1)}{2} + col] \gets value$}
                \State{\Return{}}
            \EndFunction{}
        \end{algorithmic}
        \caption{Put the value in this matrix}
    \end{algorithm}

    \pagebreak

    \textbf{(3)} Do Exercise C-3.22 of the textbook.
    \\

    Check the sizes of this two circularly linked list. If they are not in the same size,
    they obviously do not contain the same list of nodes. If they are in the same size, 
    find the same node first, then check for the rest of the nodes.
    
    \begin{algorithm}[]
        \begin{algorithmic}[1]
            \Function{Check-Two-Circularly-Linked-Lists}{$L,M$}
                \If{$Lsize$ is equal to $Msize$}
                    \State{$i \gets 0$}
                    \While{$i \neq Lsize$}
                        \If{$LCursorElem$ is equal to $MCursorElem$}
                            \If{$the~rest~of~nodes~are~the~same$}
                                \If{$n \neq 0$}
                                    \State{\Return{true}}
                                \EndIf
                            \EndIf
                            \State{$LCursor \gets LCursorNext$}
                        \EndIf
                        \State{$i \gets i + 1$}
                    \EndWhile
                \EndIf
                \State{\Return{false}}
            \EndFunction{}
        \end{algorithmic}
        \caption{Check whether the two circularly linked lists are same or not}
    \end{algorithm}

    The time complexity of this algorithm is $O(n^2)$ since it also need a loop to check
    whether the rest of nodes are the same. Note that \alg{if $n \neq 0$} is to prevent
    the situation that they contain the same starting points.
    \\

    \pagebreak

    \textbf{(4)} Do Exercise C-3.18 of the textbook using either C/C++ or pseudo code.

    \begin{lstlisting}[caption=Rearrange integer array by recursion.]
    void swap(int *a, int *b)
    {
        int tmp = *a;
        *a = *b;
        *b = tmp;
        return;
    }

    void rearrage(int *array, int start, int end)
    {
        while (array[start] % 2 == 0) { start++; }
        while (array[end] % 2 != 0) { end--; }

        if (start >= end) return;

        swap(&array[start], &array[end]);
        rearrage(array, start, end);

        return;
    }
    \end{lstlisting}

    \textbf{(5)} Do Exercise C-3.18 of the textbook, but use one single loop instead of 
    recursion.

    \begin{lstlisting}[caption=Rearrange integer array by one single loop]
    void rearrage(int *array, int start, int end)
    {
        while (start < end)
        {
            if (array[start] % 2 != 0)
            {
                if (array[end] % 2 == 0)
                {
                    swap(&array[start], &array[end]);
                    start++;
                    end--;
                }
                else { end--; }
            }
            else { start++; }
        }
        return;
    }
    \end{lstlisting}

\end{homeworkProblem}

\pagebreak

\begin{homeworkProblem}{Asymptotic Complexity}
    \textbf{(1)} Do Exercise R-4.24 of the textbook, under the assumption that both
    $d(n) - e(n) \geq 0$ and $f(n) - g(n) \geq 0$.
    \\

    Consider $d(n) = 2n^2$ and $e(n) = n^2$. Note that both $d(n)$ and $e(n)$ are $O(n^2)$,
    $d(n) - e(n) \geq 0$, and $f(n) - g(n) = n^2 - n^2 = 0 \geq 0$

    \[
        \begin{split}
            d(n) - e(n) &= n^2
            \\
            f(n) - g(n) &= 0
        \end{split}
    \]

    We can easily find that $d(n) - e(n)$ is $O(n^2)$ not $O(f(n) - g(n))$. 
    \\

    Consider another case, $d(n) = 2n^3$ and $e(n) = n^2$. Note that $d(n)$ is $O(n^3)$
    , $e(n)$ is $O(n^2)$, $d(n) - e(n) \geq 0$, and $f(n) - g(n) \geq 0$

    \[
        \begin{split}
            d(n) - e(n) &= 2n^3 - n^2
            \\
            f(n) - g(n) &= n^3 - n^2
        \end{split}
    \]

    We can easily find that $d(n) - e(n)$ is $O(n^3)$, which is also $O(f(n) - g(n))$.
    \\

    Thus, We can conclude that $d(n) - e(n)$ is not necessarily $O(f(n) - g(n))$.
    \\

    \textbf{(2)} Do Exercise R-4.29 of the textbook.
    \\

    By the big-Oh definition, we need to find a real constant $c > 0$ and
    an integer constant $n_{0} \geq 1$ such that $(n + 1)^5 \leq cn^5$ for every 
    integer $n \geq n_{0}$. Take $n_{0} = 1$, for every $n \geq n_{0}$, we have

    \[
        \begin{split}
            (n + 1)^5 &= 
            \\
            &= n^5 + 5n^4 + 10n^3 + 10n^2 + 5n^1 + 1
            \\
            &\leq 10n^5 + 10n^5 + 10n^5 + 10n^5 + 10n^5
            \\
            &= 50n^5
        \end{split}
    \]

    A possible choice is $c = 50$ and $n_{0} = 1$. Thus, we conclude that $(n + 1)^5$
    is $O(n^5)$.
    \\

    \textbf{(3)} Do Exercise R-4.39 of the textbook.
    \\

    Consider $f(n) = n \log n + 97.5n$ and $g(n) = n^2$. Note that $f(n)$ is $O(n\log n)$,
    and $g(n)$ is $O(n^2)$. For every integer \(n\), \(1 \leq n \leq 97\) we can easily
    find that 

    \[
        \begin{split}
            f(n) &=
            \\
            &= n \log n + 97.5n
            \\
            &= n \cdot (\log n + 97.5)
            \\
            &> n \cdot 97.5
            \\
            &> n^2
            \\
            &= g(n)
        \end{split}
    \]

    \pagebreak

    When $n = 98$,

    \[
        \begin{split}
            f(98) &= 98 \cdot (\log 98 + 97.5)
            \\
            &> 98 \cdot (0.5 + 97.5)
            \\
            &= 98^2
            \\
            &= g(98)
        \end{split}
    \]

    When $n = 99$,

    \[
        \begin{split}
            f(99) &= 99 \cdot (\log 99 + 97.5)
            \\
            &> 99 \cdot (1.5 + 97.5)
            \\
            &= 99^2
            \\
            &= g(99)
        \end{split}
    \]

    When $n \geq 100$,

    \[
        \begin{split}
            f(n) &=
            \\
            &= n \log n + 97.5n
            \\
            &= n \cdot (\log n + 97.5)
            \\
            &< n^2
            \\
            &= g(n)
        \end{split}
    \]

    We found the possible functions satisfying the statements from question. There is 
    $97.5n$ in $f(n)$, which isn't negligible when $n < 100$. When $n$ is getting larger,
    this term would become negligible. Therefore, A1 and Bob find the result that
    $n < 100$, the $O(n^2)$-time algorithm runs faster, and when $n \geq 100$ is the
    $O(n \log n)$-time one better.

\end{homeworkProblem}

\begin{homeworkProblem}{Playing with Big Data}

    \textbf{(1)} Describe your design of the data structure. 
    Emphasize on why you think the data structure would be (time-wise) efficient 
    for the four desired actions.
    \\

    I use Data Structure 2 from HW2 Hints. I create two 2-D Array.  \\
    First Array for \alg{Get}, \alg{Clicked}, \alg{impressed},
    \begin{center}
        \begin{tabular}{ | l | l |}
            \hline
            UserID & vector$<$data$>$\\ \hline
            0 & data with u = 0... \\ \hline
            1 & data with u = 1... \\ \hline
            \vdots & \vdots \\ \hline
            23907634 & data with u = 23907634... \\ 
            \hline
        \end{tabular}
    \end{center}

    Second Array for \alg{Profit}, 

    \begin{center}
        \begin{tabular}{ | l | l |}
            \hline
            AdID & vector$<$data$>$ (Click, Impression, UserID)\\ \hline
            0 & data with a = 0... \\ \hline
            1 & data with a = 1... \\ \hline
            \vdots & \vdots \\ \hline
           22238287& data with a = 22238287... \\ 
            \hline
        \end{tabular}
    \end{center}

\end{homeworkProblem}

\end{document}
